%--------------------------------------------------------------------
%  Capítulo 6 — Conclusiones y Trabajo Futuro
%--------------------------------------------------------------------
\chapter{Conclusiones y Trabajo Futuro}
\label{chap:conclusiones}
\justifying

Este capítulo final presenta las conclusiones derivadas del desarrollo e implementación del proyecto \textbf{Tutor Virtual}. Se realiza una recapitulación de los objetivos planteados, se evalúa el grado de consecución de los mismos, se discuten las principales aportaciones y limitaciones del trabajo realizado, y finalmente, se proponen diversas líneas de trabajo futuro que podrían dar continuidad y expandir el potencial de esta iniciativa.

%--------------------------------------------------------------------
\section{Conclusiones Generales del Proyecto}
\label{sec:conclusiones_generales}

El presente Trabajo de Fin de Grado ha abordado el diseño e implementación de un prototipo funcional de \textbf{Tutor Virtual}, un sistema de aprendizaje inteligente que integra un modelo de lenguaje grande (LLM) con ejecución local y capacidades de Recuperación Aumentada por Generación (RAG). El objetivo principal ha sido explorar la viabilidad de un tutor que combine personalización, retroalimentación formativa y un enfoque en la privacidad del usuario, al mismo tiempo que ofrece al estudiante control sobre la dificultad de los contenidos.

Las principales aportaciones de este proyecto se pueden resumir en:
\begin{itemize}
    \item \textbf{Arquitectura técnica funcional:} Se ha desarrollado una arquitectura multicapa que incluye un frontend interactivo (React), un backend robusto (FastAPI) y un motor de IA local contenerizado (Llama-3 y OpenWebUI), demostrando la integración de estas tecnologías para un sistema de tutoría.
    \item \textbf{Implementación de funcionalidades clave:} El prototipo incluye gestión de usuarios, cursos, asignaturas y temas; generación de ejercicios adaptados (teóricamente) en dificultad; y un sistema para proporcionar retroalimentación basada en IA.
    \item \textbf{Énfasis en la privacidad y autonomía del estudiante:} La ejecución local del LLM y la opción de selección manual de dificultad son características distintivas que buscan responder a preocupaciones actuales en el ámbito de la IA educativa.
    \item \textbf{Base para futuras investigaciones y desarrollos:} El sistema actual, aunque un prototipo, sienta una base sólida sobre la cual se pueden construir funcionalidades más avanzadas y realizar evaluaciones pedagógicas rigurosas.
\end{itemize}

Si bien la evaluación empírica detallada queda como trabajo futuro, el desarrollo realizado y la discusión teórica presentada en el Capítulo~\ref{chap:evaluacion_discusion} sugieren que el enfoque adoptado es prometedor y tiene el potencial de contribuir al campo de las tecnologías aplicadas a la educación.
% Comentario: Adaptar según las conclusiones reales del proyecto.

%--------------------------------------------------------------------
\section{Cumplimiento de los Objetivos Planteados}
\label{sec:conclusiones_cumplimiento_objetivos}

A continuación, se evalúa el grado de cumplimiento de los objetivos específicos establecidos al inicio del proyecto (detallados en el Capítulo~\ref{chap:introduccion}, Sección~\ref{ssec:objetivos-especificos}):
% Comentario: Asegurarse que las referencias a secciones y capítulos sean correctas.

\begin{itemize}
    \item \textbf{Objetivo Específico 1: Adaptación de contenidos guiada por el estudiante.}
    % Comentario: Evaluar si se cumplió. Ej: 
    Se ha cumplido parcialmente. El sistema implementa un selector de dificultad (fácil, intermedio, difícil) que el estudiante puede controlar explícitamente. Sin embargo, la generación de sugerencias automáticas basadas en el historial del estudiante es una funcionalidad que queda pendiente de desarrollo completo, aunque la infraestructura de datos (ej. \texttt{user\_theme\_progress}) la facilitaría.

    \item \textbf{Objetivo Específico 2: Retroalimentación formativa.}
    % Comentario: Evaluar si se cumplió. Ej:
    Se ha cumplido en su concepción técnica. El motor de IA local está diseñado para generar no solo la corrección de los ejercicios, sino también explicaciones detalladas. La calidad y efectividad pedagógica de esta retroalimentación requeriría una evaluación específica con usuarios y expertos en la materia.

    \item \textbf{Objetivo Específico 3: Analíticas de aprendizaje.}
    % Comentario: Evaluar si se cumplió. Ej:
    Se ha cumplido parcialmente. Se ha implementado un panel de progreso para estudiantes (Sección~\ref{sssec:desarrollo_dashboard_page}) que muestra métricas clave como progreso, tasa de aciertos y esfuerzo (tiempo invertido). Sin embargo, funcionalidades más avanzadas de analítica para docentes o la personalización profunda basada en estas métricas no se han abordado en esta fase.
    
    % Comentario: Añadir más objetivos específicos si se definieron y evaluar su cumplimiento.
\end{itemize}

En términos generales, se considera que los objetivos fundamentales para la creación de un prototipo funcional se han alcanzado, sentando las bases para la validación y expansión futura de las funcionalidades más avanzadas.

%--------------------------------------------------------------------
\section{Limitaciones Principales del Trabajo Realizado}
\label{sec:conclusiones_limitaciones}

Es importante reconocer las limitaciones inherentes a este proyecto, derivadas tanto de restricciones de tiempo y recursos como de la complejidad del dominio:

\begin{itemize}
    \item \textbf{Evaluación Empírica Limitada:} La principal limitación es la ausencia de una evaluación exhaustiva con usuarios finales y pruebas pedagógicas rigurosas. Los resultados presentados (Capítulo~\ref{chap:evaluacion_discusion}) se basan en una evaluación técnica y una discusión teórica del potencial del sistema. Sin datos empíricos robustos, las conclusiones sobre la efectividad del aprendizaje y la usabilidad real son preliminares.
    \item \textbf{Calidad y Variedad del Contenido Generado por IA:} Aunque se utiliza un modelo LLM avanzado, la calidad, coherencia, y adecuación pedagógica de los ejercicios y explicaciones generadas dependen fuertemente del \emph{prompt engineering}, del corpus utilizado para RAG, y de las propias capacidades (y limitaciones) del LLM. No se ha realizado una validación sistemática de este contenido por expertos.
    \item \textbf{Escalabilidad del Motor de IA Local:} Si bien la ejecución local es una ventaja para la privacidad, la escalabilidad para un gran número de usuarios simultáneos con hardware de consumo diverso podría ser un desafío. El rendimiento puede variar significativamente entre diferentes dispositivos.
    \item \textbf{Alcance Funcional del Prototipo:} Algunas funcionalidades deseables, como la adaptación automática de la dificultad basada en el rendimiento, un sistema de gamificación más elaborado, o herramientas analíticas avanzadas para docentes, no se incluyeron en el alcance de este prototipo por limitaciones de tiempo.
    \item \textbf{Contenido Pedagógico Restringido:} El sistema se desarrolló y probó inicialmente con un corpus limitado a un dominio específico (ej. matemáticas básicas). La generalización a otras asignaturas o niveles educativos requeriría la creación y adaptación de nuevos corpus y, posiblemente, ajustes en los \emph{prompts}.
    \item \textbf{Cumplimiento Normativo (RGPD):} Aunque el diseño local favorece la privacidad, una implementación en un entorno real requeriría un análisis más profundo y la implementación de mecanismos explícitos para la gestión del consentimiento, acceso, rectificación y supresión de datos, conforme al RGPD.
\end{itemize}

%--------------------------------------------------------------------
\section{Líneas de Trabajo Futuro y Mejoras Potenciales}
\label{sec:conclusiones_trabajo_futuro}

El prototipo actual de Tutor Virtual abre numerosas vías para futuras investigaciones y desarrollos. A continuación, se proponen algunas de las más relevantes:

\begin{enumerate}[label=\arabic*., leftmargin=*]
    \item \textbf{Evaluación Rigurosa con Usuarios y Pruebas Pedagógicas:} Realizar estudios controlados con estudiantes para medir el impacto en el aprendizaje (utilizando pre-tests y post-tests), la usabilidad (mediante cuestionarios estandarizados como SUS y pruebas de tareas) y la satisfacción general. Recopilar feedback cualitativo detallado.
    \item \textbf{Mejora Continua del Motor de IA y Contenidos:}
        \begin{itemize}
            \item Experimentar con diferentes LLMs locales (nuevos modelos, diferentes tamaños y técnicas de cuantización) para optimizar el equilibrio entre rendimiento y calidad de generación.
            \item Refinar las técnicas de \emph{prompt engineering} y expandir/curar los corpus documentales para RAG para mejorar la calidad y diversidad de los ejercicios y explicaciones.
            \item Implementar mecanismos de validación de contenido generado por IA, posiblemente con la ayuda de expertos en la materia.
        \end{itemize}
    \item \textbf{Desarrollo de Adaptatividad Avanzada:} Implementar algoritmos de \emph{knowledge tracing} o modelos bayesianos para inferir el nivel de conocimiento del estudiante y ofrecer una adaptación de la dificultad más dinámica y automática, complementando la selección manual.
    \item \textbf{Ampliación del Banco de Contenidos y Dominios:} Extender el sistema para soportar múltiples asignaturas, niveles educativos y tipos de contenido (ej. preguntas de opción múltiple, problemas de programación, etc.).
    \item \textbf{Mejoras en la Interfaz y Experiencia de Usuario (UX):}
        \begin{itemize}
            \item Incorporar elementos de gamificación (puntos, insignias, rankings) para aumentar la motivación y el compromiso.
            \item Desarrollar un panel de control más completo para docentes, con herramientas para supervisar el progreso de los estudiantes y gestionar contenidos.
            \item Refinar la interfaz basándose en el feedback de las pruebas de usabilidad.
        \end{itemize}
    \item \textbf{Implementación de Funcionalidades de Colaboración:} Explorar la posibilidad de que los estudiantes puedan interactuar entre sí, o que los docentes puedan crear grupos y asignar tareas.
    \item \textbf{Optimización del Rendimiento y Escalabilidad:} Investigar técnicas para reducir el tiempo de \emph{warm-up} del LLM local y optimizar el uso de recursos. Para despliegues a mayor escala, considerar arquitecturas híbridas o estrategias de balanceo de carga si se opta por servidores con GPUs.
    \item \textbf{Cumplimiento Normativo y Ético Exhaustivo:} Desarrollar un módulo completo para la gestión del consentimiento RGPD, y realizar una auditoría ética del sistema, especialmente en lo referente a los sesgos potenciales en la IA y la equidad.
    \item \textbf{Integración con Plataformas LMS Existentes:} Explorar la posibilidad de integrar Tutor Virtual con Sistemas de Gestión del Aprendizaje (LMS) populares mediante estándares como LTI (Learning Tools Interoperability).
    \item \textbf{Investigación en Interpretabilidad (XAI):} Desarrollar mecanismos para que el sistema pueda ofrecer justificaciones más claras sobre cómo el LLM llega a ciertas explicaciones o evaluaciones, aumentando la transparencia y la confianza del usuario.
\end{enumerate}

%--------------------------------------------------------------------
\section{Conclusión Final}
\label{sec:conclusiones_final}

En conclusión, el proyecto Tutor Virtual ha logrado desarrollar un prototipo que demuestra la viabilidad de un sistema de tutoría inteligente personalizado, con ejecución local de IA y control de dificultad por parte del estudiante. A pesar de las limitaciones inherentes a un trabajo de esta naturaleza, se han sentado las bases para una herramienta con un potencial considerable para mejorar la experiencia de aprendizaje y apoyar la autonomía del estudiante.

Las aportaciones técnicas y conceptuales, junto con la identificación de claras líneas de trabajo futuro, subrayan la relevancia y el recorrido potencial de esta línea de investigación y desarrollo. Se espera que este trabajo pueda servir como punto de partida para futuras exploraciones en el campo de la inteligencia artificial aplicada a la educación, buscando siempre soluciones innovadoras, accesibles y centradas en el usuario.
