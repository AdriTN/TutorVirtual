%--------------------------------------------------------------------
%  Introducción
%--------------------------------------------------------------------

\chapter{Introducción}
\justifying

%--------------------------------------------------------------------
\section{Contexto y motivación}
\label{sec:contexto-motivacion}

\subsection*{La transformación del aprendizaje en línea}

En los últimos años, los \emph{MOOC} han pasado de ser proyectos experimentales a constituir una pieza importante de la formación digital. Esto ha sido posible gracias al cloud computing, que abate los costes de infraestructura; a la extensión de la banda ancha en Europa —que alcanza al \SI{89}{\percent} de los hogares— \cite{Eurostat2024}; y al auge de repositorios de \emph{Open Educational Resources}, que han reducido cerca de un \SI{65}{\percent} el gasto en creación de materiales didácticos \cite{UNESCO2023OER}. Sin embargo, menos del \SI{15}{\percent} de los inscritos completa sus cursos, lo que deja claro la brecha entre acceso y aprendizaje efectivo.

%--------------------------------------------------------------------
\subsection*{Principales obstáculos estructurales}

La investigación identifica dos limitaciones claras detrás de este bajo rendimiento:  
\begin{itemize}[leftmargin=*]
  \item \textbf{Falta de adaptación individualizada.} Muchos programas no consideran ni los conocimientos previos ni las preferencias de aprendizaje de cada usuario.  
  \item \textbf{Feedback diferido.} Al concentrar la evaluación al término de cada módulo, se ralentiza la detección de errores y aumenta la deserción.  
\end{itemize}
Meta-análisis recientes muestran que los Sistemas de Tutoría Inteligente bien diseñados pueden acercarse en eficacia a la tutoría presencial \cite{Kulik2016}.

%--------------------------------------------------------------------
\subsection*{La aportación de los modelos de lenguaje}

Los \emph{Large Language Models} (por ejemplo, GPT-4o o Claude 3) permiten generar explicaciones a medida, diseñar ejercicios con niveles de dificultad progresiva y corregir respuestas abiertas aportando observaciones formativas. Para incorporarlos eficazmente es importante lograr latencias inferiores a \SI{500}{ms} \cite{Kramer2024} y respetar los principios éticos de la UNESCO \cite{UNESCO2021,UNESCO2023}.

\bigskip
\noindent
A partir de estos problemas, este trabajo propone crear un \textbf{tutor inteligente basado en IA generativa} que combine las ventajas de los LLM con buenas prácticas de diseño instruccional y analíticas de aprendizaje.

%--------------------------------------------------------------------
\section{Planteamiento del problema}

\subsection*{Objetivo general}

Desarrollar e implementar un \textbf{tutor inteligente con IA generativa} capaz de ofrecer formación personalizada que contribuya al logro del ODS 4 (Educación de calidad).

\subsection*{Objetivos específicos}

\begin{itemize}[leftmargin=*]
  \item \textbf{Adaptación guiada por el estudiante}: disponer de un conjunto de ejercicios con diferentes niveles de dificultad seleccionables por el usuario.  
  \item \textbf{Comentarios inmediatos}: proporcionar retroalimentación formativa instantánea, fomentando la confianza del alumno.  
  \item \textbf{Analíticas de rendimiento}: mostrar en un panel métricas importantes (avance, esfuerzo, tasa de éxito, nivel de confianza) garantizando privacidad y transparencia de datos.
\end{itemize}

%--------------------------------------------------------------------
\section{Metodología}

La investigación seguirá un enfoque iterativo e incremental, a través de ciclos de modelado, desarrollo, pruebas y documentación. Cada etapa incluye una estimación de esfuerzo y las tareas principales asociadas.

Esto se ha comentado en entregas anteriores. Sin embargo, a lo largo de este documento se irán comentando todas esas etapas.
