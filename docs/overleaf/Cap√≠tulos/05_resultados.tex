%--------------------------------------------------------------------
%  Capítulo 5 — Evaluación y Discusión del Prototipo
%--------------------------------------------------------------------
\chapter{Evaluación y Discusión del Prototipo}
\label{chap:evaluacion_discusion}
\justifying

Este capítulo se centra en una evaluación cualitativa del prototipo de \textbf{Tutor Virtual} desarrollado, analizando cómo su diseño e implementación abordan los objetivos planteados y discutiendo su potencial impacto. Aunque no se han realizado pruebas empíricas exhaustivas con usuarios o benchmarks de rendimiento detallados en esta fase del proyecto, se examinarán las características del sistema en relación con los indicadores clave de rendimiento (KPIs) y los requisitos funcionales y no funcionales definidos previamente. Esta discusión sentará las bases para las conclusiones finales y las recomendaciones de trabajo futuro.

%--------------------------------------------------------------------
\section{Análisis del Diseño y Funcionalidades Implementadas}
\label{sec:eval_analisis_diseno}

El desarrollo de Tutor Virtual ha culminado en un prototipo que integra los componentes clave definidos en la arquitectura: un motor de IA local, un backend RESTful y una interfaz de usuario frontend interactiva. A continuación, se discuten los aspectos más relevantes de la implementación en relación con los objetivos del proyecto.

%------------------------------------------------
\subsection{Evaluación del Componente de Inteligencia Artificial Local}
\label{ssec:eval_ia_local}

La decisión de implementar un motor de IA con ejecución local (\texttt{Llama-3 8B} cuantizado vía \texttt{llama.cpp} y OpenWebUI para RAG) fue central para abordar los requisitos de privacidad (R7: Funcionamiento offline-first) y potencialmente reducir costes operativos.

\paragraph{Calidad de la Interacción (Teórica).}
La combinación de un LLM potente como Llama-3 con RAG sobre un corpus documental específico (ej. \texttt{numerosNaturales.txt}) tiene el potencial de generar ejercicios relevantes y explicaciones contextualizadas.

%------------------------------------------------
\subsection{Valoración de la Arquitectura Backend y Frontend}
\label{ssec:eval_arquitectura_be_fe}

La arquitectura de dos repositorios separados para backend (FastAPI) y frontend (React+Vite), junto con la estructura interna modular de cada uno, fue diseñada para promover la mantenibilidad y escalabilidad.

\paragraph{Backend (FastAPI).}
La elección de FastAPI, con su tipado estricto y generación automática de OpenAPI, facilita un desarrollo robusto y bien documentado de la API. El sistema de autenticación JWT y la gestión de la base de datos con SQLAlchemy y Alembic proporcionan una base sólida. Se espera que la API sea eficiente para las operaciones CRUD y la gestión de la lógica de negocio, como se describió en el Capítulo~\ref{chap:desarrollo}.

\paragraph{Frontend (React + Vite).}
El uso de React con Vite y React Query permite la creación de interfaces de usuario reactivas y eficientes. Los componentes reutilizables (\texttt{MultiCheckList}, \texttt{CrudModal}, \texttt{EntityTable}, etc.) desarrollados para el panel de administración y las vistas de estudiante buscan optimizar la experiencia de usuario y la consistencia. La gestión del estado del servidor con React Query debería minimizar las cargas innecesarias y mantener la UI sincronizada con el backend.

\paragraph{Interacción entre Capas.}
La integración mediante una API RESTful bien definida y el uso de DTOs generados a partir de OpenAPI (o definidos manualmente con consistencia) es fundamental para la correcta comunicación. Los mecanismos de autenticación y manejo de errores en el cliente (ej. interceptores de Axios) contribuyen a la robustez del sistema completo.

%--------------------------------------------------------------------
\section{Desafíos Anticipados para una Evaluación Empírica Futura}
\label{sec:eval_desafios_futura_eval}

Una evaluación empírica rigurosa del sistema Tutor Virtual, aunque esencial, presentaría ciertos desafíos:

\begin{itemize}[leftmargin=*]
    \item \textbf{Diseño Experimental Robusto:} Medir el impacto en el aprendizaje (KPI-2) requeriría un diseño experimental cuidadoso, posiblemente con grupos de control, pre-tests y post-tests validados, y una muestra de participantes representativa, lo cual implica una inversión de tiempo y recursos considerable.
    \item \textbf{Calidad del Contenido Generado por IA:} Evaluar objetivamente la calidad pedagógica, la corrección y la adecuación de los ejercicios y el feedback generados por el LLM es complejo y a menudo requiere la intervención de expertos en el dominio.
    \item \textbf{Recopilación de Datos de Uso y Privacidad:} Aunque el sistema es local, si se quisiera recopilar datos de uso anónimos para investigación (con consentimiento), se necesitaría implementar mecanismos seguros y éticos para ello.
    \item \textbf{Variabilidad del Hardware del Usuario:} El rendimiento del motor de IA local puede variar significativamente dependiendo del hardware del usuario final, lo que podría afectar la experiencia y los resultados de las pruebas de latencia si se realizan en múltiples dispositivos.
\end{itemize}
