% ===================================================================
%                       P R E Á M B U L O
% ===================================================================

%--------------------------------------------------------------------
% 1 · Idioma y tipografía
%--------------------------------------------------------------------
\usepackage[spanish]{babel}
\usepackage{fontspec}                % Compilar con XeLaTeX/LuaLaTeX
  \setmainfont{TeX Gyre Pagella}      % Serif elegante y libre

%--------------------------------------------------------------------
% 2 · Márgenes y geometría
%--------------------------------------------------------------------
\usepackage{geometry}
  \geometry{a4paper, margin=2.5cm}

%--------------------------------------------------------------------
% 3 · Colores e hipervínculos
%--------------------------------------------------------------------
\usepackage{xcolor}
\usepackage{hyperref}
  \hypersetup{
    colorlinks,
    linkcolor  = blue!60!black,
    citecolor  = blue!60!black,
    urlcolor   = blue!60!black,
    pdfauthor  = {Nombre Estudiante},
    pdftitle   = {Título del TFG}
  }

%--------------------------------------------------------------------
% 4 · Figuras y flotantes
%--------------------------------------------------------------------
\usepackage{graphicx}
  \graphicspath{{figures/}{docs/Ilustraciones/}}  % ← añade tus rutas
\usepackage{float}                                % [H] exactamente aquí

%--------------------------------------------------------------------
% 5 · Tablas y unidades
%--------------------------------------------------------------------
\usepackage{booktabs, multirow, array}
\usepackage{siunitx}
  \sisetup{detect-all, per-mode=symbol}
  \DeclareSIUnit\billion{\,B}
  \DeclareSIUnit\million{\,M}
  \DeclareSIUnit\USD{USD}

%--------------------------------------------------------------------
% 6 · Enumeraciones
%--------------------------------------------------------------------
\usepackage{enumitem}
  \setlist[enumerate]{label=\arabic*., leftmargin=*}
  \setlist[itemize]{leftmargin=*}

%--------------------------------------------------------------------
% 7 · Alineación de texto
%--------------------------------------------------------------------
\usepackage{ragged2e}   % habilita \justifying
\usepackage{booktabs}
\usepackage{tabularx}

%--------------------------------------------------------------------
% 8 · Código fuente, DBML y algoritmos
%--------------------------------------------------------------------
\usepackage{listings}
\usepackage{algorithm}
\usepackage{algpseudocode}

% --- Definición básica de DBML para listings ------------------------
\lstdefinelanguage{dbml}{
  morekeywords={Table,Ref,PK,Indexes,Enum},
  sensitive=true,
  comment=[l]{//},
  morecomment=[s]{/*}{*/},
  morestring=[b]",
}

\lstset{
  basicstyle   = \fontsize{8}{10}\ttfamily,
  captionpos   = b,
  frame        = single,
  numbers      = left,
  numberstyle  = \tiny,
  keywordstyle = \color{blue!70!black},
  commentstyle = \color{gray!70},
  breaklines   = true,
  tabsize      = 2
}

%--------------------------------------------------------------------
% 9 · Glosario
%--------------------------------------------------------------------
\usepackage{glossaries}
  \makeglossaries

%--------------------------------------------------------------------
% 10 · Bibliografía (Biber + IEEE)
%--------------------------------------------------------------------
\usepackage[
  backend  = biber,
  style    = ieee,
  language = spanish
]{biblatex}
  \addbibresource{refs.bib}

%--------------------------------------------------------------------
% 11 · Epígrafes
%--------------------------------------------------------------------
\usepackage{epigraph}

%--------------------------------------------------------------------
% 12 · Numeración de figuras y tablas por capítulo
%--------------------------------------------------------------------
\usepackage{chngcntr}
  \counterwithin{figure}{chapter}
  \counterwithin{table}{chapter}

%--------------------------------------------------------------------
% 13 · Formato de capítulos
%--------------------------------------------------------------------
\usepackage{titlesec}
  \setcounter{secnumdepth}{3}

  \titleformat{\chapter}
    {\normalfont\huge\bfseries}
    {\thechapter\ }{0pt}{}

  \renewcommand{\chaptername}{}          % elimina la palabra “Capítulo”
  \renewcommand{\thechapter}{\arabic{chapter}}

%--------------------------------------------------------------------
% 14 · Cabeceras y pies de página
%--------------------------------------------------------------------
\usepackage{fancyhdr}
  \setlength{\headheight}{15pt}
  \pagestyle{fancy}
  \fancyhf{}
  \fancyhead[R]{\thepage}
  \fancyhead[L]{\nouppercase{\leftmark}}
  \renewcommand{\headrulewidth}{0.4pt}

%--------------------------------------------------------------------
% 15 · Espaciado de párrafo
%--------------------------------------------------------------------
\setlength{\parskip}{4pt}
\setlength{\parindent}{0pt}

%--------------------------------------------------------------------
% 16 · Miscelánea
%--------------------------------------------------------------------
\usepackage{csquotes}
\usepackage{mathtools}
%--------------------------------------------------------------------
